%\documentclass[trans]{beamer}
\documentclass{beamer}

\usepackage{natim-beamer}
\usepackage{listings}

\usetheme[shownavigation={true},  % true | false
          logo={images/novapost.png},
          titlepageimage={images/titleimage.png},
          header=utbm,          % fullnav | shortnav | utbm
          dept={Recherche \& Dévelopment}
        ]{UTBM}
\usecolortheme{Terra}
\usepackage{upgreek}

\title[Un poney qui parle le russe]{Apprenez le russe à votre poney}
\subtitle{Novapost - Automne \the\year}
\author{Rémy HUBSCHER}
\institute{remy.hubscher@novapost.fr}
\date{\today}


\hypersetup{
      pdfpagemode = FullScreen,% afficher le pdf en plein écran
      pdfauthor   = {Rémy HUBSCHER},%
      pdftitle    = {Apprenez le russe à votre poney},%
      pdfsubject  = {NOVAPOST - Les applications multilingues},%
      pdfkeywords = {novapost, i18n, l10n, django},%
      pdfcreator  = {PDFLaTeX},%
      pdfproducer = {PDFLaTeX}%
}

\begin{document}
\lstset{language=python,showstringspaces=false,breaklines=true,xleftmargin=7mm,basicstyle=\scriptsize,frame=single,framexleftmargin=7mm,captionpos=b,tabsize=2,numbers=left,numberstyle=\scriptsize,escapechar={\%}{\_}{\^},literate={è}{{\`e}}1 {é}{{\'e}}1 {à}{{\`a}}1 {ñ}{{\~n}}1 {ç}{{\c c}}1 {й}{{selectfontchar233}}1,}
\selectlanguage{french}

\begin{frame}[plain]
  \titlepage
\end{frame}

% \AtBeginSection[]{
%    \begin{frame}
%      \frametitle{Sommaire}
%      %%% affiche en début de chaque section, les noms de sections et
%      %%% noms de sous-sections de la section en cours.
%      \tableofcontents[currentsection,hideothersubsections]
%    \end{frame} 
% }

\begin{frame}
  \frametitle{Sommaire}
  \tableofcontents[pausesections,hideothersubsections]
\end{frame}


\section{Une application multilingue ?}

\begin{frame}
  \frametitle{Une application multilingue ?}
  \begin{exampleblock}{Qu'est-ce que c'est ?}
    \begin{enumerate}
      \pause \item Une application qui semble avoir été développée pour la langue natale de l'utilisateur
      \pause \item Toutes les interactions avec l'utilisateur sont faites dans sa langue
    \end{enumerate}
  \end{exampleblock}
    \pause
  \begin{alertblock}{Quand ?}
    \begin{enumerate}
      \pause \item C'est important de se poser la question
      \pause \item Faut-il une même application traduite ? ou une installation de l'application pour chaque langue ?
      \pause \item Les contenus seront-il absolument identiques dans toutes les langues ?
    \end{enumerate}
  \end{alertblock}
\end{frame}

\section{Comment traduire le code source ?}

\begin{frame}[fragile]
  \frametitle{Traduire le code source}
    \begin{alertblock}{Comment ?}
    \begin{enumerate}
      \pause \item Django permet de traduire le contenu avec gettext
      \pause \item Dans le code Python ainsi que dans les templates
    \end{enumerate}
    \end{alertblock}
    \pause
    \textbf{Les vues}
  \begin{lstlisting}
from django.utils.translation import ugettext as _

def ma_super_vue(request):
    return HttpResponse(_('Hello'))
  \end{lstlisting}
  \pause
  \textbf{Les templates}
  \begin{lstlisting}


  \end{lstlisting}

\end{frame}


\begin{frame}[fragile]
  \frametitle{Traduire le code source}

\begin{enumerate}
  \item À certains endroits, le code est exécuté au lancement de l'application.
  \pause \item À ce moment, on ne connaît pas encore la langue de l'utilisateur
  \pause \item On utilise alors \texttt{ugettext\_lazy} afin que le texte soit remplacé non pas à l'exécution, mais à l'affichage.
\end{enumerate}
\pause
  \textbf{Les models}
  \begin{lstlisting}
from django.utils.translation import ugettext_lazy as _

class Category(models.Model):
    slug = models.SlugField(_('slug'))
    name = models.CharField(_('name'), max_length=40)
    description = models.TextField(_('description'), blank=True)

    class Meta:
        verbose_name = _('category')
        verbose_name_plural = _('categories')
  \end{lstlisting}

\end{frame}

\begin{frame}[fragile]
  \frametitle{Configurer Django}
  \begin{block}{Activer le mode multilingue de Django}
    \pause
  \begin{lstlisting}
# -*- coding: utf-8 -*-

# load the internationalization machinery.
USE_I18N = True

# will not format dates, numbers and calendars according
# to the current locale.
USE_L10N = True

# will not use timezone-aware datetimes.
USE_TZ = True

ugettext = lambda s: s
LANGUAGES = (
    ('fr', u'Français'),
    ('es', u'Español'),
    ('en', u'English'),
    ('de', u'Deutch'),
)
  \end{lstlisting}


  \end{block}

\end{frame}

\begin{frame}[fragile]
  \begin{exampleblock}{Récupérer les textes à traduire}
    \begin{enumerate}
      \pause \item La commande \texttt{makemessages} permet de récupérer les textes à traduire
      \pause \item On créé un répertoire \texttt{locale} pour le projet ou pour l'app
      \pause \item \texttt{python manage.py makemessages -l fr --no-location}
      \pause \item \texttt{\$EDITOR locale/fr/LC\_MESSAGES/django.po}
    \end{enumerate}
  \end{exampleblock}
  \pause

  \begin{lstlisting}
msgid "Hello"
msgstr "Bonjour"
  \end{lstlisting}

\pause
On utilise \texttt{python manage.py compilemessages} pour compiler le \texttt{django.po} en catalogue gettext \texttt{django.mo}

\end{frame}

\begin{frame}[fragile]
  \begin{alertblock}{Pour aller plus loin}
    \begin{enumerate}
      \pause \item La gestion des pluriels
      \pause \item Le contexte de traduction
      \pause \item Les variables de substitution
    \end{enumerate}
  \end{alertblock}
  \pause

  \begin{lstlisting}


  Request created the {{ date }} at {{ time }}

  \end{lstlisting}
\pause

  \begin{lstlisting}
from django.utils.translation import ungettext, ugettext

page = ungettext('there is %(count)d object',
                 'there are %(count)d objects',
                 count) % {'count': count}

ugettext('mon texte et la variable %s' % variable)
ugettext('mon texte et la variable %s') % variable
  \end{lstlisting}

\end{frame}

\section{Traduction des URL}

\begin{frame}[fragile]
  \frametitle{Traduire les URL}
  \vspace{-0.7em}
  \begin{alertblock}{Traduire les URL}
      \begin{enumerate}
         \pause \item Si vous n'en voyez pas l'intérêt ne le faites pas
         \pause \item Il faut s'assurer que deux URL ne se chevauchent pas
      \end{enumerate}
  \end{alertblock}
\pause

  \begin{lstlisting}
from django.conf.urls import patterns, include, url
from django.utils.translation import ugettext_lazy as _
category_patterns = patterns(''
    url(r'^$', 'category.views.list'),
    url(r'^(?P<slug>[\w-]+)/$', 'category.views.detail'))
ihm_parrterns = patterns('',
    url(_(r'^about/$'), 'about.view'),

    url(_(r'^categories/'), include(category_patterns,
        namespace='category')))
urlpatterns = patterns('',
    url(r'^sitemap\.xml$', 'sitemap.view'),
    url(_(r'^en/'), include(ihm_patterns)),)
  \end{lstlisting}
\pause
Traduire \texttt{\textasciicircum en/} avec le code de langue dans chaque langue.

\end{frame}

\section{Vérifier ses fichiers de traductions}
\begin{frame}
  \frametitle{Vérifier ses fichiers de traductions}

  \begin{enumerate}
    \item On souhaite vérifier s'il y a des phrases non traduites
    \pause \item \texttt{pip install check\_po}
    \pause \item \texttt{check\_po locale/fr/LC\_MESSAGES/django.po}
    \pause \item On souhaite vérifier que toutes les URL sont bien uniques
    \pause \item \texttt{check\_urls locale/fr/LC\_MESSAGES/django.po}
  \end{enumerate}
\end{frame}

\section{Outils de gestion du multilingue}

\begin{frame}
  \frametitle{Des outils de sélection de la langue et des url}
  \begin{enumerate}
    \item Des middlewares de sélection automatique de la langue (Préfixe / Utilisateur / Session / Navigateur / Défaut...)
    \pause \item Des templatetags utiles \pause \texttt{\{\% current\_page "fr" \%\}}
    \pause \item \texttt{pip install django-i18nurl}
  \end{enumerate}

\end{frame}

\begin{frame}[fragile]
  \frametitle{Sélecteur de la langue}
  \begin{enumerate}
    \item \texttt{\{\% get\_available\_languages as LANGUAGES \%\}}
    \pause \item \texttt{\{\% for lang\_code, lang\_name in LANGUAGES \%\}}
    \pause \item \texttt{\{\% language lang\_code \%\} \{\% endlanguage \%\}}
    \pause \item \texttt{\{\% current\_i18nurl lang\_code \%\}}
  \end{enumerate}
\pause
\vfill
  Exemple complet : \url{https://gist.github.com/Natim/6722268}
\vfill
\end{frame}

\begin{frame}[fragile]
  \frametitle{Traduire les models}
  \begin{enumerate}
    \item Plusieurs solutions
    \pause \item \textbf{django-nani} Avec une table pour stocker les traductions
    \pause \item \textbf{django-modeltranslation} Avec un champ par langue
    \pause \item On n'ajoute pas des langues sans modifier le code
    \pause \item \textbf{django-modeltranslation} avec \textbf{South} c'est cool
  \end{enumerate}
\pause
\textbf{category/translation.py}
\begin{lstlisting}
from modeltranslation.translator import translator, TranslationOptions
from category.models import Category

class CategoryTranslationOptions(TranslationOptions):
    fields = ('name', 'slug', 'description')

translator.register(Category, CategoryTranslationOptions)
\end{lstlisting}
\pause
\texttt{from modeltranslation.admin import TranslationAdmin}
\end{frame}



\begin{frame}[fragile]
  \frametitle{Gérer les mails}
  \vspace{-0.7em}
  \begin{enumerate}
    \item Même sans le multilingue il faut bien gérer ses mails
    \pause \item Avec le multilingue, on peut avoir des images qui changent
    \pause \item \textbf{django-mail-factory} est une bonne solution
    \pause \item On écrit ses mails dans les templates
    \pause \item On a un chemin permettant de faire un template spécifique dans une langue
    \pause \item \texttt{/admin/mails} permet de tester ses mails et de les prévisualiser
    \pause \item On peut mettre des images inline et des pièces jointes
  \end{enumerate}
\pause
\textbf{category/mails.py}
\begin{lstlisting}
from mail_factory import factory, BaseMail

class WelcomeEmail(BaseMail):
    template_name = 'activation_email'
    params = ['user', 'site_name', 'site_url']

factory.register(WelcomeEmail)
\end{lstlisting}
\end{frame}



\begin{frame}
  \frametitle{Conclusion}

  \begin{enumerate}
     \item Django possède des outils (Internationalisation, Localisation, Timezone)
     \pause \item Novapost a développé des outils (\textbf{check\_po}, \textbf{django-i18nurl}, \textbf{django-mail-factory})
     \pause \item Faire une application complètement multilingue qui semble native \pause \textbf{C'EST POSSIBLE}
     \pause \item Pour les CMS, faites juste des catégories liées à une langue et le contenu interne dans une seule langue.
  \end{enumerate}

\end{frame}

\begin{frame}
  \frametitle{Questions ?}
  \vfill
  \Large\centerline{Merci pour votre attention}
  \pause
  \Large\centerline{Avez vous des questions ?}
  \vfill
  \image{django.png}{0.6}
\end{frame}

\end{document}
